\documentclass[12pt, a4paper, final]{article}

\usepackage{cmap}
\usepackage[T2A]{fontenc}
\usepackage{mathtext}
\usepackage[utf8]{inputenc}
\usepackage[english,russian]{babel}
\usepackage[left=2cm,right=2cm,top=2cm,bottom=2cm]{geometry}
\usepackage{graphicx}
\usepackage{color}
\usepackage{float}
\usepackage{wrapfig}
\usepackage{pgfplots, pgfplotstable}
\pgfplotsset{compat=1.9}

\usepackage{amsfonts}
\usepackage{amsmath}
\usepackage{amssymb}
\usepackage{amsthm}
\usepackage{mathtools}
\usepackage{cancel}
\usepackage{mathrsfs}
\usepackage{icomma}
\mathtoolsset{showonlyrefs=true}
\pagestyle{empty}

\begin{document}

\begin{center}

    \textsc{\LARGE Московский Физико-Технический Институт}\\[1,5cm]
    \textsc{\Large Параллельное программирование}\\[0,5cm]
    \textsc{\large Лабораторная работа \textnumero 1}\\[1cm]

    \noindent\rule{\textwidth}{1pt}
    \\[0.5cm]
    { \huge \bfseries Уравнение переноса}
    \\[0.1cm]
    \noindent\rule{\textwidth}{1pt}
\end{center}

\vspace{1cm}

В данной работе предлагается ознакомиться с реализацией явной центральной трехточечной схемы для нахождения приближенного решения уравнения переноса:
\begin{equation*}
\begin{cases}
\dfrac{\partial u(t,x)}{\partial t} + A \cdot \dfrac{\partial u(t, x)}{\partial x} = f(t, x)\\
u(0, x) = \varphi(x)\\
u(t, 0) = \psi(t)\\
u(t, X) = \psi'(t)
\end{cases}
~~~~~
\begin{cases}
0 \leq x \leq X = 1\\
0 \leq t \leq T = 1
\end{cases}
\end{equation*}

В нашем случае разностная схема будет иметь следующий вид:
\begin{equation*}
\dfrac{u_m^{k+1} - 0.5 (u_{m + 1}^k + u_{m - 1}^k)}{\tau} + A \cdot \dfrac{u_{m + 1}^k - u_{m - 1}^k}{2h} = f_m^k
\end{equation*}
\begin{equation*}
\begin{cases}
t = k \tau\\
x = m h
\end{cases}
~~~~~
\begin{cases}
0 \leq k \leq K = 40\\
0 \leq m \leq M = 20
\end{cases}
~~~~~
\begin{cases}
T = K \tau\\
X = M h
\end{cases}
~~~~~
A = 0.5
\end{equation*}

Начальные условия --- $ \varphi(x), \psi(t), \psi'(t) $ --- для задачи подобраны так, чтобы они удовлетворяли уравнению $ u(t, x) = \sin(\pi (t + 0.5)) \sin(\pi x) $.\\

Программа выводит данные (таблицу) для построения графиков: значения функции $ u(t, x) = \sin(\pi (t + 0.5)) \sin(\pi x) $ на выбранной сетке и результат работы явной центральной трехточечной схемы.\\

Распараллеливание реализовано для подсчета каждого последующего слоя по времени согласно стандарту MPI.

\begin{quote}
Данные приведены для случая $ K = 40, M = 20 $ в следствие ограниченности ресурсов среды \LaTeX.
\end{quote}

\vfill

\begin{minipage}[b]{0.33\textwidth}
    \textit{Выполнил:}\\
    Р.Р. Валиев, 715 гр.\\\\
    \textit{Проверил:}\\
    А.С. Герасимов
\end{minipage}

\newpage

{\centering \subsection*{Искомая функция для сравнения с результатами работы.}}

\begin{center}
\begin{tikzpicture}
\begin{axis}[
view={210}{10},
xmin=0,
xmax=1,
ymin=0,
ymax=1,
grid=both, xlabel={x}, ylabel={t}, zlabel={$ u(x, t) = \sin(\pi (t + 0.5)) \sin(\pi x) $}, title={\text{Искомая функция}}]
\addplot3[surf,samples=50,domain=0:1]
{sin(180 * (y + 0.5)) * sin(180 * x)};
\end{axis}
\end{tikzpicture}
\end{center}

{\centering \subsection*{Сравнение значений для оценки результата работы.}}

\begin{tikzpicture}
\begin{axis}[view={120}{10}, grid=both, xlabel={k}, ylabel={m}, title={\text{Точные значения}}]
\addplot3[surf, mesh/rows=41] file {DATA_EXACT.dat};
\end{axis}
\end{tikzpicture}
\begin{tikzpicture}
\begin{axis}[view={120}{10}, grid=both, xlabel={k}, ylabel={m}, title={\text{Результат работы ЯС}}]
\addplot3[surf, mesh/rows=41] file {DATA.dat};
\end{axis}
\end{tikzpicture}

\end{document}
